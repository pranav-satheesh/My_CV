%%%%%%%%%%%%%%%%%%%%%%%%%%%%%%%%%%%%%%%%%
% Medium Length Professional CV
% LaTeX Template
% Version 2.0 (8/5/13)
%
% This template has been downloaded from:
% http://www.LaTeXTemplates.com
%
% Original author:
% Trey Hunner (http://www.treyhunner.com/)
%
% Important note:
% This template requires the resume.cls file to be in the same directory as the
% .tex file. The resume.cls file provides the resume style used for structuring the
% document.
%
%%%%%%%%%%%%%%%%%%%%%%%%%%%%%%%%%%%%%%%%%

%----------------------------------------------------------------------------------------
%	PACKAGES AND OTHER DOCUMENT CONFIGURATIONS
%----------------------------------------------------------------------------------------

\documentclass{resume} % Use the custom resume.cls style
\usepackage[dvipsnames]{xcolor}
\definecolor{royalpurple}{rgb}{0.47, 0.32, 0.66}

\usepackage[left=0.75in,top=0.6in,right=0.75in,bottom=0.6in]{geometry} % Document margins
\newcommand{\tab}[1]{\hspace{.2667\textwidth}\rlap{#1}}
\newcommand{\itab}[1]{\hspace{0em}\rlap{#1}}
\name{Pranav Satheesh} % Your name
%\address{556 North Shore Drive 1 \\ South Haven, MI 49090} % Your address
%\address{123 Pleasant Lane \\ City, State 12345} % Your secondary addess (optional)
% \address{github: loisks317 \\ blog: loisks.blogspot.com}

\address{Department of Physics, University of Florida}
\address{\href{mailto:pranavsatheesh@ufl.edu}{pranavsatheesh@ufl.edu}
 \\ \href{https://pranav-satheesh.github.io}{Website} \\ \href{https://github.com/pranav-satheesh}{Github} } % Your phone number and email


%\usepackage{fancyhdr}
% \pagestyle{fancy}
% \fancyhf{} % Clear the header/footer
% \fancyfoot[L]{\textit{Last updated on: \today}} % Date on the left side of the footer
% \fancyfoot[C]{\thepage} % Page number in the center of the footer


% \renewcommand{\headrulewidth}{0pt} % Remove the top line

\usepackage{fancyhdr}
\pagestyle{fancy}
\renewcommand{\headrulewidth}{0pt}
\fancyhf{}
\fancyfoot[CE,CO]{\footnotesize Pranav Satheesh ~\textbullet~ Curriculum Vitae}
\fancyfoot[LE,LO]{\footnotesize Mar 2025}
\fancyfoot[RE,RO]{\footnotesize \thepage}


%\usepackage[svgnames]{xcolor}
% \definecolor{C2}{RGB}{137, 6, 32}
% \definecolor{C1}{RGB}{0, 109, 119}
% \definecolor{C3}{RGB}{8, 76, 97}


\definecolor{C2}{RGB}{54, 79, 107}  % Steel Blue
\definecolor{C1}{RGB}{123, 104, 238} % Medium Slate Blue
\definecolor{C3}{RGB}{72, 61, 139}  % Dark Slate Blue


\renewenvironment{rSection}[1]{
\sectionskip
\textbf{\textcolor{C2}{\MakeUppercase{#1}}}
\sectionlineskip
\hrule
\begin{list}{}{
\setlength{\leftmargin}{1.5em}
}
\item[]
}{
\end{list}
}

\newcommand{\conference}[1]{\textcolor{C2}{\textbf{#1}}}


\usepackage[colorlinks = true,
            linkcolor = C3,
            urlcolor  = C3,
            citecolor = C3,
            anchorcolor = C3]{hyperref}

\begin{document}

%----------------------------------------------------------------------------------------
%	EDUCATION SECTION
%----------------------------------------------------------------------------------------
\begin{rSection}{Research Interests}
Massive black hole dynamics, Galaxy-black hole co-evolution, Massive black hole formation, Gravitational Waves

\end{rSection}

\begin{rSection}{Education}

\textbf{University of Florida} \hfill 2022 - Present \\
\emph{Ph.D. in Physics}\\
Advisor: Dr. Laura Blecha

\textbf{Indian Institute of Technology Madras} \hfill 2017 - 2022 \\
\emph{BS-MS in Physics}

\end{rSection}

\begin{rSection}{Appointments}

\textbf{Graduate Research Assistant
} \hfill 2023 - Present \\
University of Florida \\
Department of Physics 

\textbf{Graduate Teaching Assistant
} \hfill 2022 - 2023 \\
University of Florida \\
Department of Physics 

\end{rSection}

\begin{rSection}{Awards and scholarships} \itemsep -2pt 
{{\color{C3} Steigleman fellowship} for graduate research in theoretical astrophysics in UF Physics} \hfill {\em 2024}\\
{{\color{C3} Graduate fellowship} for first-year graduate students from UF Physics}\hfill {\em 2022} \\
{{\color{C3} 62nd Institute day} award for academic performance in Physics from IIT Madras} \hfill {\em 2021} \\
{Selected for \href{https://swissnex.org/india/thinkswiss/}{ThinkSwiss Research Scholarship}} \hfill {\em 2020}\\
{Recipient of the \href{http://www.inspire-dst.gov.in/scholarship.html}{INSPIRE-DST Scholarship for Higher Education}} \hfill {\em 2017 - 2022}
\end{rSection}


\begin{rSection}{Publications}
\begin{itemize}
    \item \textbf{Satheesh, P.}, Blecha, L., \& Kelley, L. Z. (2025). \emph{Mergers and Recoil in Triple Massive Black Hole Systems from Illustris}, (in prep.)
    \item Paul, K., Maurya, A., Henry, Q., Sharma, K., \textbf{Satheesh, P.}, Divyajyoti, Kumar, P., \& Mishra, C. K. (2024). \emph{ESIGMAHM: An Eccentric, Spinning inspiral-merger-ringdown waveform model with Higher Modes for the detection and characterization of binary black holes}, \href{https://arxiv.org/abs/2409.13866}{arXiv:2409.13866}.
\end{itemize}

\end{rSection}

\begin{rSection}{Presentations}
    \begin{rSubsection}{\textit{Talks}}{}{}{}
        \item \textbf{Satheesh, P.}, Blecha, L. \emph{Studying gravitational wave and slingshot recoil of massive black holes in cosmological simulations}, \conference{33rd Midwest Relativity meeting, University of Chicago}, November 2023 
    \end{rSubsection}
    \begin{rSubsection}{\textit{Posters}}{}{}{}
        \item \textbf{Satheesh, P.}, Blecha, L., \& Kelley, L. \emph{Mergers, recoil and slingshot in triple massive black hole systems}, \conference{The era of binary supermassive black holes, Aspen Center for Physics} , February 2025
        \item \textbf{Satheesh, P.}, Blecha, L., \& Kelley, L. \emph{Studying gravitational wave and slingshot recoil of massive black holes in cosmological simulations}, \conference{Nanograv Fall collaboration meeting, University of Michigan}, October 2024
        \item \textbf{Satheesh, P.}, Gandhi, S., \& Mishra, C.K. \emph{Fisher analysis of eccentric binaries with higher mode frequency domain inspirals}, \conference{LIGO-Virgo-KAGRA collaboration meeting (virtual)},  March 2022 
        \item \textbf{Satheesh, P.}, Saha, P., Schmid, H. \emph{A spectropolarimetric method for predicting the gravitational wave polarization of LISA verification binaries}, \conference{237th American Astronomical Society meeting}, 2021
    \end{rSubsection}
\end{rSection}

\begin{rSection}{Professional Memberships}
{Associate Member,  {\color{C3} Nanograv collaboration}} \hfill {\em 2024 - \rm Present}\\
{Graduate Member,  {\color{C3} American Astronomical Society}} \hfill {\em 2023 - \rm Present}\\
{Graduate Member, {\color{C3} American Physical Society}}\hfill {\em 2022-2023} \\
{Member, {\color{C3} LIGO Scientific Collaboration}} \hfill {\em 2021 - 2022}
\end{rSection}


% \begin{rSection}{Experience}

% \begin{rSubsection}{Insight Data Science}{June 2016 - August 2016}{Fellow}{}
% \item Developed a program to analyze Fitness Tracker data and compare results with weather information from Weather Underground, providing user with an analysis how to optimize their activity based on past habits.
% \item Wrote an API to web scrape Polar Loop Fitness Tracker data 
% \item Data was obtained with Selenium, Data was managed through SQL, Pandas and Scipy were used for analysis, and Matplotlib was used for visualizations
% \end{rSubsection}


%------------------------------------------------

% \begin{rSubsection}{NSF Graduate Research Fellow}{September 2013 - September 2016}{Graduate Reseach Assistant }{}
% \item Analyzed multiple time series satellite data sets ($>$ 1 Tb of data) to explore low energy ion loss in the inner Plasmasphere 
% \item Developed an analytic model to demonstrate loss of ions from increased wave activity
% \item Developed algorithm to properly account for variability in the low energy ion pitch angle measurements when calculating partial ion densities. 
% \end{rSubsection}

% \end{rSection}


%	EXAMPLE SECTION
%----------------------------------------------------------------------------------------


\begin{rSection}{Technical Skills }

\begin{tabular}{ @{} >{\bfseries}l @{\hspace{6ex}} l }
Programming Languages &  Python, C/C++ \\
Software \& Tools &  Mathematica, LaTeX, Git, GitHub, Shell scripting, hpg\\
High Performance Computing    & Slurm, OpenMP \\
Python Packages & Pandas, Matplotlib, Numpy, Scipy, Astropy, \\
  &Scikit-learn \\
Operating Systems   &  Linux, Windows, macOS \\
\end{tabular}

\end{rSection}

\begin{rSection}{Teaching
Experience}

Teaching Assistant, \textbf{Code-Astro 2023} \hfill Summer 2023 \\
\emph{Taught sessions at this software engineering workshop for astronomy \\ supported by
the Heising-Simons Foundation.}

Teaching Assistant, \textbf{Physics-1 with calculus lab} \hfill Fall 2022 \\
\emph{Introductory laboratory course for undergraduates taking Physics I \\
with calculus at the University of Florida.}

Grader, \textbf{Complex Networks} \hfill Spring 2021 \\
\emph{Graduate-level course at IIT Madras.}


\end{rSection}


\begin{rSection}{Science communication and outreach}
    \begin{rSubsection}{Service}{}{}{}
    \item Organizer, {\color{C3} \textbf{Comscicon Flagship Workshop}} \hfill 2025
    \item Writer and Illustrator, {\color{C3} \textbf{ComSciWri}} \hfill 2025 - Present
    \item Author, {\color{C3} \textbf{Astrobites}}\hfill 2023-2025
        % Astrobites is a science-communication collaboration of graduate students from around the world working to
        % summarise daily papers in astronomy in a bite-sized accessible format.
    \item Coordinator, {\color{C3} \textbf{Physics Graduate Community}}\hfill 2023-Present
    \item Head, {\color{C3} \textbf{Horizon: The Physics and Astronomy Club of IIT Madras}} \hfill 2019-2020
     % I headed the student run physics and astronomy club at IIT Madras under the Center of Innovation (CFI). I played an integral role in engaging the student community through various projects, lectures, workshops, and competitive events.
    \end{rSubsection}

    \begin{rSubsection}{Science-communication pieces}{}{}{}
    %\item Comic on Feedback 
    \item \href{https://www.comscicon.org/news/origin-story-supermassive-black-holes}{The Origin Story of Supermassive Black Holes}, science comic featured in \\
    ComSciConversation Blog
    \item \href{https://sciwri.club/archives/13482}{The origin of supermassive black holes}, science comic featured in Club SciWri
    \item Science articles featured in \href{https://aasnova.org/?s=Pranav+Satheesh+}{AAS Nova}
    \item \href{https://astrobites.org/author/psatheesh/}{Articles} written by as an author for Astrobites. 

    \end{rSubsection}

    \begin{rSubsection}{Talks}{}{}{}
    \item \textit{\color{C2} Scripting the origin story of supermassive black holes}, talk given as a part \\
    of Graduate Student and Postdoc Seminar series at UF physics \hfill Sep 2024
     \item \textit{\color{C2} Python for Astronomy}, An \href{https://youtu.be/HfYR0uwYAyM}{Youtube lecture series} offered by me as part of Horizon \hfill Jul 2020
    \item \textit{\color{C2} Relativity and Gravitation}, \href{https://github.com/HorizonIITM/summer-school-2021}{Horizon-IITM Summer School} \hfill July 2021
    \item \textit{\color{C2} Analysis of Globular Clusters Using
    Colour-Magnitude Diagrams}, Shaastra IITM \hfill Jan 2020
    \end{rSubsection}
\end{rSection}


\begin{rSection}{Workshops Attended }
    {Nanograv student workshop, Michigan} \hfill {\em 2024}\\
    {Comscicon flagship workshop, Boston} \hfill {\em 2024}\\
    {ICTS Summer School on Gravitational Wave Astronomy} \hfill {\em2023}\\
    {School on Black Holes and Gravitational Waves, Chennai, India} \hfill {\em 2022}\\
    {Physics of the Early Universe, ICTS (Online)} \hfill {\em 2020}
    
 %    \item \href{https://youtu.be/zXDrQ_-WNUg}{\textbf{LISC Continous Gravitational Wave Workshop}} (Online) \hfill Oct 2021
 %    \item \href{https://sites.psu.edu/paxvii/}{\textbf{Physics and Astrophysics at the Extreme (PAX-VII) Workshop}} (Online) \hfill Aug 2021
 %    %\item Participant, \textbf{2021 Sagan Exoplanet Summer Virtual Workshop}, NASA Exoplanet Science Institute, California Institute of Technology, July 2021 
	% \item \href{https://www.icts.res.in/program/gws2021}{\textbf{ICTS Summer School on Gravitational Wave Astronomy}} (Online) \hfill Jul 2021 
	% \item \href{http://ipta4gw.org/meetings/2021/}{\textbf{IPTA Student Workshop}} (Online) \hfill June 2021 
	% \item \href{https://www.icts.res.in/program/peu}{\textbf{Physics of the Early Universe}}, ICTS (Online)\hfill  Sep 2020 
	% \item \href{https://www.icts.res.in/program/gws2020}{\textbf{ICTS Summer School on Gravitational Wave Astrophysics}} \hfill May 2020 
\end{rSection}

\end{document}
